\setcounter{chapter}{0}
\chapter{Εισαγωγή}\label{introduction}

\chapterprecis{
Η εισαγωγική ενότητα περιγράφει τον στόχο της διπλωματικής εργασίας. Επίσης, καθορίζει τα όρια και το πεδίο εφαρμογής της. Τέλος, παρουσιάζει το κίνητρο πίσω από αυτήν την ερευνητική εργασία.}

\section{Διατύπωση Προβλήματος}\label{problem-statement}

Η αεροελαστικότητα είναι ένας κλάδος της φυσικής και της μηχανικής που μελετά την απόκριση ελαστικών σωμάτων όταν εκτίθενται σε ροή ρευστού. Οι δυνάμεις που εμπλέκονται σε αυτήν την αλληλεπίδραση είναι η αδρανειακή, η ελαστική και η αεροδυναμική. Τα αεροελαστικά προβλήματα στη μηχανική μπορούν να ταξινομηθούν σε δύο βασικές κατηγορίες. Την στατική αεροελαστικότητα, η οποία εξετάζει την απόκριση σε κατάσταση μόνιμης ροής, και τη δυναμική αεροελαστικότητα, η οποία επικεντρώνεται κυρίως στη δονητική απόκριση του σώματος.

Τα πιο συνηθισμένα αεροελαστικά φαινόμενα που παρατηρούνται στα αεροσκάφη είναι:

\begin{itemize}
\item
  \emph{\textlatin{Aerodynamic Divergence}}, όπου η εκτροπή των ανυψωτικών επιφανειών ενός αεροσκάφους προκαλεί επιπλέον άνωση, η οποία με τη σειρά της οδηγεί σε περαιτέρω εκτροπή προς την ίδια κατεύθυνση, με αποτέλεσμα την εμφάνιση υπερβολικών τάσεων ή ακόμη και την αστοχία της κατασκευής.
\item
  \emph{\textlatin{Aeroelastic Control reversal}}, όπου οι δυνάμεις που παράγονται από το πηδάλιο ελέγχου (το οποίο είναι υπεύθυνο για τον έλεγχο του αεροσκάφους στον διαμήκη του άξονα) είναι αρκετές για να προκαλέσουν στρέβλωση της πτέρυγας σε τέτοιο βαθμό ώστε να αλλάξουν τα χαρακτηριστικά άντωσης της πτέρυγας, καθιστώντας τις επιφάνειες ελέγχου αναποτελεσματικές ή ακόμη και παράγοντας το αντίθετο από το επιθυμητό αποτέλεσμα.
\item
  \emph{\textlatin{Aeroelastic Flutter}} είναι μια δυναμική αστάθεια μιας κατασκευής που προκύπτει λόγω της αλληλεπίδρασης της ροής του ρευστού με τις ιδιομορφές της κατασκευής.
\end{itemize}

Στην παρούσα διπλωματική εργασία θα μελετηθούν τα χαρακτηριστικά ταλάντωσης (\textlatin{flutter}) μιας ανυψωτικής επιφάνειας και θα προσαρμοστούν σε συγκεκριμένες απαιτήσεις με τη χρήση τεχνικών βελτιστοποίησης.


\section{Στόχος}\label{objective}

Ο στόχος αυτού του έργου είναι να αναπτυχθεί κατανόηση των χαρακτηριστικών του αεροελαστικού πτερυγισμού μιας ανυψωτικής επιφάνειας κατασκευασμένης από σύνθετο υλικό και των μεθόδων που χρησιμοποιούνται για την υπολογιστική πρόβλεψη της περιοχής πτερυγισμού χρησιμοποιώντας τον επιλυτή \textlatin{Optistruct} της \textlatin{Altair}. Στη συνέχεια, μελετάται η επίδραση καποιων κατασκευαστικών παραμέτρων για να προσδιοριστεί η επίδρασή τους στα χαρακτηριστικά του πτερυγισμού, και εξερευνείται η αποτελεσματικότητα διαφόρων τεχνικών βελτιστοποίησης για την προσαρμογή των χαρακτηριστικών του πτερυγισμού σε ένα σύνολο απαιτήσεων με τη χρήση \textlatin{Python}.

\section{Πεδίο Εφαρμογής και Περιορισμοί}\label{scope-and-limitations}

Το έργο θα περιλαμβάνει:

\begin{itemize} 
  \item Μελέτη των σύνθετων υλικών και της υλοποίησής τους στον επιλυτή \textlatin{Optistruct}.
  \item Μελέτη της μεθόδου \textlatin{Vortex Lattice} και της χρήσης της κατά την ανάλυση αεροδυναμικού πτερυγισμού. 
  \item Μελέτη της ανάλυσης αεροελαστικού πτερυγισμού σε θεωρητικό επίπεδο μέσω των εξισώσεων κίνησης και σε υπολογιστικό επίπεδο μελετώντας την εφαρμογή της θεωρίας σε εμπορικούς επιλυτές. 
  \item Διερεύνηση της επίδρασης των ιδιοτήτων των σύνθετων υλικών στα χαρακτηριστικά του πτερυγισμού. 
  \item Διερεύνηση διαφόρων τεχνικών βελτιστοποίησης για την προσαρμογή των χαρακτηριστικών του πτερυγισμού. Οι τεχνικές αυτές περιλαμβάνουν \textlatin{line search} μεθόδους, γενετικούς αλγόριθμους και νευρωνικά δίκτυα. 
\end{itemize}

\section{Κίνητρο}\label{motivation}

Στη βιομηχανία της αεροδιαστημικής, η ελαχιστοποίηση του βάρους των κατασκευών έχει εξαιρετική σημασία, καθώς επιτρέπει την αύξηση του χρήσιμου φορτίου, βελτιώνει την αποδοτικότητα, τη ευελιξία και άλλα χαρακτηριστικά ελέγχου. Μία συνέπεια της ελαχιστοποίησης του βάρους των κατασκευών είναι η μείωση των συντελεστών ασφαλείας σε σύγκριση με άλλους τομείς της μηχανικής, γεγονός που δημιουργεί την ανάγκη για πιο ακριβείς υπολογισμούς για να εξασφαλιστεί η ασφάλεια.

Καθώς το κόστος των δοκιμών πρωτοτύπων στον τομέα της αεροδιαστημικής παραμένει υψηλό, οι υπολογιστικές προσομοιώσεις χρησιμοποιούνται ολοένα και περισσότερο και εξελίσουν τις δυνατότητές τους να μελετούν ευρύτερο φάσμα φαινομένων. Η παρούσα διπλωματική διατριβή θα βοηθήσει φοιτητές και τους ερευνητές να κατανοήσουν τις βασικές αρχές της ανάλυσης ευστάθειας των πτερύγων, \textlatin{(wing flutter)}, και θα αναπτύξει τη διαδικασία βελτιστοποίησης για τέτοιες κατασκευές. Ο κώδικας που θα αναπτυχθεί σε αυτή την έργασία μπορεί επίσης να επεκταθεί και να τροποποιηθεί από οποιονδήποτε αναλάβει μια παρόμοια έργασία.
