\chapter{Συμπεράσματα \& Μελλοντική Εργασία}
\label{conclusions-future-work}
\chapterprecis{Σε αυτό το κεφάλαιο παρουσιάζεται μια περίληψη των αποτελεσμάτων, εξάγονται γενικά συμπεράσματα και προτείνονται ιδέες για περαιτέρω ανάπτυξη.}

\section{Βελτιστοποίηση}\label{optimization}

Στο παρόν κεφάλαιο, παρουσιάζεται μια περίληψη των αποτελεσμάτων που παράχθηκαν σε αυτή την εργασία και εξάγονται γενικά συμπεράσματα επ' αυτών.

Στον \autoref{tab:flutter_results} παρατίθεται μια περίληψη των αποτελεσμάτων που αποκτήθηκαν με την εκάστοτε μέθοδο βελτιστοποίησης που χρησιμοποιήθηκε στην παρούσα εργασία, καθώς και ένας πίνακας, \autoref{tab:flutter_changes}, με την ποσοστιαία διαφορά από την αρχική ανάλυση.

\selectlanguage{english}
\begin{table}[h]
  \footnotesize
  \centering
  \setlength{\tabcolsep}{5pt}
  \renewcommand{\arraystretch}{1.2}
  \begin{tabularx}{\textwidth}{lccc}
      \toprule
      \textbf{Method} & \textbf{Flutter Velocity (m/s)} & \textbf{Mass (Kg)} & \textbf{Divergent Mode} \\
      \midrule
      Initial Flutter Characteristics & 94.11 & 67.33 & 3 \\
      Powell’s Method Scenario 1 & 99.48 & 67.33 & 3 \\
      Powell’s Method Scenario 2 & 175.28 & 67.33 & 3 \\
      Genetic Algorithm & 214.69 & 45.1 & 1 \\
      \bottomrule
  \end{tabularx}
  \caption{\textgreek{Σύνοψη των αποτελεσμάτων βελτιστοποίησης}}
  \label{tab:flutter_results}
\end{table}

\begin{table}[h]
  \centering
  \footnotesize
  \setlength{\tabcolsep}{5pt}
  \renewcommand{\arraystretch}{1.2}
  \begin{tabular}{lcc}
      \toprule
      \textbf{Method} & \textbf{Velocity Change (\%)} & \textbf{Mass Change (\%)} \\
      \midrule
      Powell’s Method Scenario 1 & 5.71\% & 0.00\% \\
      Powell’s Method Scenario 2 & 86.25\% & 0.00\% \\
      Genetic Algorithm & 128.13\% & -33.02\% \\
      \bottomrule
  \end{tabular}
  \caption{\textgreek{Σύγκριση των αποτελεσμάτων των μεθόδων βελτιστοποίησης}}
  \label{tab:flutter_changes}
\end{table}
\selectlanguage{greek}

Αυτό που είναι σαφές από τη σύνοψη των αποτελεσμάτων είναι ότι υπάρχει σημαντικός χώρος για βελτίωση σε σχέση με την αρχική λύση. 
Η επίδραση των γωνιών των στρώσεων είναι αξιοσημείωτη και κάτι που πρέπει να λαμβάνεται υπ' όψιν κατά το σχεδιασμό ενός φτερού αεροπλάνου με \textlatin{composite laminate} υλικά.
Αυτού του είδους η βελτιστοποίηση αξίζει να πραγματοποιείται, διότι μπορεί ενδεχομένως να μειώσει τη μάζα του αεροσκάφους, με αποτέλεσμα βελτιωμένες επιδόσεις πτήσης, χειρισμό και οικονομία καυσίμου.
Αξιοσημείωτο είναι, ωστόσο, ότι η δυναμική αστάθεια πτερυγισμού δεν είναι ο μόνος περιοριστικός παράγοντας για τη δομή της κύριας πτέρυγας και υπάρχουν άλλοι παράγοντες που πρέπει να ληφθούν υπ' όψιν. 
Ένας κύριος εξ αυτών είναι η παραμόρφωση της πτέρυγας και η δυναμική απόκριση υπό διαφορετικά φορτία και γωνίες προσβολής.
Ένας άλλος παράγοντας είναι τα φαινόμενα στατικής αεροελαστικότητας και η αντιστροφή του ελέγχου πτήσης.

Από τον \autoref{tab:flutter_changes} φαίνεται ότι το καλύτερο αποτέλεσμα επιτυγχάνεται με τη χρήση του γενετικού αλγορίθμου, ο οποίος καταφέρνει να μειώσει τη μάζα της πτέρυγας κατά
ένα διόλου αμελητέο $33\%$, ενώ ταυτόχρονα καταφέρνει να αυξήσει την ταχύτητα πτερυγισμού κατά $128\%$.
Αυτό το εντυπωσιακό αποτέλεσμα πρέπει να αντιμετωπιστεί με πολύ προσοχή, 
διότι όπως έχει συζητηθεί στην \autoref{genetic-algorithm-optimization}, η λύση φαίνεται
να καθίσταται ασταθής νωρίτερα, αν και το πρόσημο της απόσβεσης της πρώτης ιδιομορφής δεν αλλάζει μέχρι τα $214 m/s$.

\section{Πρόβλεψη με Νευρωνικά Δίκτυα}
\label{neural-network-prediction.}

Από την ανάπτυξη Νευρωνικών Δικτύων για την πρόβλεψη της ταχύτητας πτερυγισμού αυτής της γεωμετρίας, 
καταλήξαμε στο συμπέρασμα ότι απαιτείται ένα αρκετά περίπλοκο νευρωνικό δίκτυο 
για μία ικανοποιητική ακρίβεια στις προβλέψεις.

Επιπρόσθετα, τα δεδομένα που απαιτούνται για την εκπαίδευση του νευρωνικού, 
είναι υπολογιστικά ασύμφορο να αποκτηθούν και, 
στις περισσότερες περιπτώσεις, η υπολογιστική πολυπλοκότητα για τη δημιουργία 
των δεδομένων εκπαίδευσης ξεπερνά κατά πολύ
την αντίστοιχη για την πραγματοποίηση μιας άμεσης βελτιστοποίησης.

Η ακρίβεια των προβλεπόμενων αποτελεσμάτων είναι αρκετά καλή στις περισσότερες περιπτώσεις, 
αλλά όχι αρκετή για ακριβή πρόβλεψη πτερυγισμού σε κάθε περίπτωση, 
διότι υπάρχουν μερικά \textlatin{outliers} στα δεδομένα όπου η πρόβλεψη δεν είναι ικανοποιητική.

Αξιοσημείωτο είναι ότι μετά την εκπαίδευση του μοντέλου, απαιτείται σχεδόν μηδενικό
υπολογιστικό κόστος για να πραγματοποιηθεί μια πρόβλεψη σχετικά με την ταχύτητα πτερυγισμού. 
Το Νευρωνικό Δίκτυο μπορεί στη συνέχεια να χρησιμοποιηθεί για τη βελτιστοποίηση της κατασκευής, αν και η ακρίβεια των αποτελεσμάτων δεν είναι εγγυημένη.


\section{Μελλοντική Εργασία}\label{future-work}

Η παρούσα μελέτη παρέχει κάποιες εφαρμοσμένες πρακτικές και ερμηνείες στην ανάλυση πτερυγισμού και τις τεχνικές 
προσαρμογής πτερυγισμού χρησιμοποιώντας μεθόδους βελτιστοποίησης και \textlatin{Python}, αλλά 
παραμένουν αρκετές ανεξερεύνητες περιοχές που θα μπορούσαν να διαλευκανθούν με περαιτέρω μελέτη.

Οι μελλοντικές προσπάθειες θα πρέπει να περιλαμβάνουν τουλάχιστον μια πειραματική 
επικύρωση των αποτελεσμάτων της προσομοίωσης. Με αυτόν τον τρόπο, η επίδραση των 
γωνιών των στρώσεων του σύνθετου υλικού μπορεί να επιβεβαιωθεί στην πράξη.

Αν και πολλοί αλγόριθμοι βελτιστοποίησης χρησιμοποιήθηκαν σε αυτή τη μελέτη με 
υποσχόμενα αποτελέσματα, παραμένουν ανεξερεύνητες πολλές προηγμένες τεχνικές. 
Κάποια μελλοντική εργασία θα μπορούσε να εξετάσει αυτούς τους πιο πολυδιάστατους 
αλγορίθμους, καθώς και την ενδεχόμενη εφαρμογή μεθόδων βασισμένων στη μηχανική μάθηση
για περαιτέρω ενίσχυση των διαδικασιών βελτιστοποίησης.

Το φαινόμενο του πτερυγισμού που μελετήθηκε σε αυτή τη διατριβή, μοντελοποιημένο με 
την αεροδυναμική θεωρία \textlatin{Vortex Lattice} σε συνδυασμό με έναν επιλυτή 
βασισμένο στην ανάλυση ιδιομορφών, θα μπορούσε επίσης να προσομοιωθεί μέσω μιας 
\textlatin{transient} ανάλυσης αλληλεπίδρασης Ρευστού-Κατασκευής \textlatin{(FSI)}. Αυτό θα επέτρεπε μια 
σύγκριση των αποτελεσμάτων μεταξύ των δύο προσεγγίσεων, παρέχοντας περαιτέρω 
επικύρωση και σύγκριση των αποτελεσμάτων διαφορετικών μεθόδων.

Τέλος, η ενσωμάτωση των Νευρωνικών Δικτύων Ενημερωμένων από Φυσική \textlatin{(PINNs)} 
σε αυτή την εφαρμογή, πρόκειται για μια υποσχόμενη κατεύθυνση για μελλοντική έρευνα. 
Τα \textlatin{PINNs} είναι ένας αναδυόμενος κλάδος της υπολογιστικής μηχανικής, που συνδυάζει την εκμάθηση 
βασισμένη σε δεδομένα με τους φυσικούς νόμους που διέπουν την εκάστοτε εφαρμογή, όπως οι εξισώσεις 
της ροής ρευστού και της δομικής δυναμικής. Ενσωματώνοντας τη φυσική άμεσα στη 
διαδικασία εκπαίδευσης του νευρωνικού δικτύου, τα \textlatin{PINNs} μπορούν να προσφέρουν 
μια πιο αποτελεσματική και ακριβή λύση σε περίπλοκα προβλήματα αλληλεπίδρασης 
ρευστού-κατασκευής. Η δυνατότητά τους να παρακάμψουν κάποιους από τους περιορισμούς 
των παραδοσιακών αριθμητικών μεθόδων, όπως η δημιουργία πλέγματος και τα 
ζητήματα σύγκλισης του επιλυτή, τα καθιστά ιδιαίτερα ελκυστικά για τη μελέτη μηχανικών συστημάτων.

