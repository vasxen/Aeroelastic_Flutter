\setcounter{chapter}{0}
\chapter{Introduction}\label{introduction}

\chapterprecis{
The Introduction chapter describes the aim of the thesis. It also
defines the limits and scope of this work. Finally, it describes the
motivation behind this project.}

\section{Problem Statement}\label{problem-statement}

Aeroelasticity is a branch of physics and engineering which studies the
response of elastic bodies exposed to a fluid flow. The forces involved
in this interaction are inertial, elastic and aerodynamic. Aeroelastic
problems in engineering can be classified into two broad categories:
static aeroelasticity which deals with the steady state response, and
dynamic aeroelasticity dealing mainly with the body's vibrational
response.

The most common aeroelastic effects encountered by aircraft are:

\begin{itemize}
\item
  \emph{Aerodynamic Divergence}, where the deflection of lifting
  surfaces of an aircraft leads to additional lift that, in turn, leads
  to further deflection in the same direction, resulting in excessive
  stress or even leading to structural failure
\item
  \emph{Aeroelastic Control reversal,} where the forces generated by the
  control aileron (responsible for the roll control of an aircraft) are
  sufficient to twist the wing itself to such an extent that it changes
  the lift characteristics of the wing which makes the control surfaces
  ineffective or even produces the opposite of the expected result
\item
  \emph{Aeroelastic Flutter} is a dynamic instability of a structure
  that occurs due to the interaction of the fluid flow with the
  eigenmodes of the structure
\end{itemize}

In this thesis the flutter characteristics of a lifting surface will be
explored and tailored to specific requirements using optimization
techniques.

\section{Objective}\label{objective}

The objective of this project is to develop an understanding of the
flutter characteristics of a lifting surface made of a laminate
composite material and the methods used to computationally predict the
flutter region using Altair's Optistruct solver. Thereafter the effect
of several structural parameters is studied to determine their effect on
the flutter characteristics and the effectiveness of several
optimization techniques is tested to tailor the flutter characteristics
to a set of requirements using Python.

\section{Scope and Limitations}\label{scope-and-limitations}

The project will include:

\begin{itemize}
\item
  The study of composite materials and their implementation in the
  Optistruct solver
\item
  The study of the Vortex Lattice panel Method and how it is used during
  the Aerodynamic Flutter Analysis
\item
  The study of Aeroelastic Flutter Analysis on a theoretical basis by
  means of the governing equations of motion and on a computational
  basis by studying the implementation of theory into commercial
  solvers.
\item
  Investigation of the effect of composite material properties on the
  flutter characteristics
\item
  Investigation of different optimization techniques to manipulate the
  flutter characteristics including line search methods, genetic
  algorithms and Neural Networks
\end{itemize}

\section{Motivation}\label{motivation}

In the aerospace industry minimization of weight of structures is of
utmost importance, since it allows the increase of useful payload,
improves efficiency maneuverability and other control characteristics. A
consequence of structural weight minimization is the reduction of safety
margins in comparison to other engineering fields, thus the need for
more precise calculations arises to ensure safety.

As aerospace prototype testing costs are elevated, computer simulations
are used more and more and become ever more advanced with the ability to
study a wider range of phenomena. This project would help students and
researchers understand the basics of wing flutter instability and
provide insight into the optimization process for such structures. The
code developed in this project could also be extended and modified by
anyone tackling a similar project.
