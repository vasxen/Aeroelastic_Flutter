\appendix
\chapter{Python Code for Optimization}
\label{appendix}

In this appendix the main code for the simplest case of optimization
using Powell's method is presented. The code for all the other
applications is modified but the core logic and many parts remain the
same throughout. All the code including the code used to process the
results and produce the graphics in this thesis is available on GitHub:
\href{https://github.com/vasxen/Aeroelastic_Flutter}{vasxen/Aeroelastic\_Flutter
(github.com)}.
% Define VS Code-style colors
\definecolor{vscode_bg}{rgb}{0.16, 0.16, 0.16}      % Dark gray background
\definecolor{vscode_keyword}{rgb}{0.86, 0.36, 0.07} % Orange keywords
\definecolor{vscode_string}{rgb}{0.67, 0.83, 0.27}  % Green strings
\definecolor{vscode_comment}{rgb}{0.5, 0.5, 0.5}    % Gray comments
\definecolor{vscode_funcname}{rgb}{0.25, 0.72, 0.85} % Blue function names
\definecolor{vscode_number}{rgb}{0.79, 0.43, 0.07}  % Orange numbers

% Define VS Code-style Python listing
\lstdefinestyle{vscodepython}{
    language=Python,
    backgroundcolor=\color{white}, 
    basicstyle=\ttfamily\footnotesize\color{black}, % White text
    keywordstyle=\color{vscode_keyword}\bfseries, 
    stringstyle=\color{vscode_string}, 
    commentstyle=\color{vscode_comment}\itshape, 
    numberstyle=\color{vscode_number},
    identifierstyle=\color{black}, % Default text color for variables
    emph={self}, emphstyle=\color{vscode_funcname}, % Highlight "self" in blue
    frame=single, 
    rulecolor=\color{vscode_bg}, 
    numbers=left, 
    numbersep=5pt, 
    stepnumber=1, 
    breaklines=true, 
    showstringspaces=false, 
    tabsize=4
}
\lstinputlisting[style=vscodepython, caption={Python script for data processing}, label={lst:python_script}]{C:/Users/vasxen/OneDrive/Thesis/code/optimization.py}