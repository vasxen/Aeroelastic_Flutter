\chapter{Conclusions \& Future Work}
\label{conclusions-future-work}
\chapterprecis{In this chapter a summary of the results is presented, more gerenal conclusions are drawn and ideas for further development are discussed}

\section{Optimization }\label{optimization}

A summary of the results produced in this work is now made and general
conclusions are drawn.

in \autoref{tab:flutter_results} is a summary of the results obtained with every different
optimization method used in this work, as well as a table with the
percentage difference from the initial flutter solution.

\begin{table}[h]
  \footnotesize
  \centering
  \setlength{\tabcolsep}{5pt}
  \renewcommand{\arraystretch}{1.2}
  \begin{tabularx}{\textwidth}{lccc}
      \toprule
      \textbf{Method} & \textbf{Flutter Velocity (m/s)} & \textbf{Mass (Kg)} & \textbf{Divergent Mode} \\
      \midrule
      Initial Flutter Characteristics & 94.11 & 67.33 & 3 \\
      Powell’s Method Scenario 1 & 99.48 & 67.33 & 3 \\
      Powell’s Method Scenario 2 & 175.28 & 67.33 & 3 \\
      Genetic Algorithm & 214.69 & 45.1 & 1 \\
      \bottomrule
  \end{tabularx}
  \caption{Summary of Optimization Methods}
  \label{tab:flutter_results}
\end{table}

\begin{table}[h]
  \centering
  \footnotesize
  \setlength{\tabcolsep}{5pt}
  \renewcommand{\arraystretch}{1.2}
  \begin{tabular}{lcc}
      \toprule
      \textbf{Method} & \textbf{Velocity Change (\%)} & \textbf{Mass Change (\%)} \\
      \midrule
      Powell’s Method Scenario 1 & 5.71\% & 0.00\% \\
      Powell’s Method Scenario 2 & 86.25\% & 0.00\% \\
      Genetic Algorithm & 128.13\% & -33.02\% \\
      \bottomrule
  \end{tabular}
  \caption{Comparison of different optimization methods}
  \label{tab:flutter_changes}
\end{table}

What is clear from the results summary is that there is a lot of room
for improvement from the initial solution. The effect of the ply angles
is considerable and something that should be considered when designing a
composite laminate aircraft wing. This kind of optimization is worth
executing because it can potentially reduce the structural mass of the
aircraft which results in enhanced flight performance, handling and fuel
economy. It is worth noting though that dynamic flutter instability is
not the only limiting factor of a main wing structure and there are
other factors that should be considered. One of the main considerations
being the wing deformation and dynamic response under different loads
and angles of attack. Another consideration is static aeroelastic
effects and flight control reversal.

From \autoref{tab:flutter_changes} it can be seen that the best result is achieved while using
the genetic algorithm which manages to reduce the mass of the wing by a
very substantial 33\% while still managing to increase the flutter speed
by 128\%. This impressive result should be treated with a lot of
caution, because as it has been discussed in \autoref{genetic-algorithm-optimization} the solution
seems to become unstable earlier, even though the sign of the damping of
the first mode does not change until 214 m/s.

\section{Neural Network prediction}
\label{neural-network-prediction.}

From the development of Neural Networks to predict the flutter speed of
this structure we have concluded that quite a complex neural network is
required for the best accuracy in predictions. What is more, the data
required for training is very computationally expensive to acquire and,
in most cases, the computational effort of generating the training data
far exceeds the computation effort required for a direct optimization.

The accuracy of the predicted results is good enough in most cases but
not enough for accurate flutter prediction in every case because there
are some outliers in the data, where the prediction is not satisfactory.

It is worth noting that after the model has been trained, there is
almost no computational effort required to make a prediction about the
flutter speed. The Neural network can then be used for optimization of
the structure.

\section{Future Work}\label{future-work}

The current study provides some insight into flutter analysis and
flutter tailoring techniques using optimization methods and python, but
several avenues remain unexplored and could benefit from further
investigation.

Future efforts should include at least some experimental validation of
the simulation results. This way the influence of the ply-angles of the
laminate composite material could be verified in practice

Although several optimization algorithms were employed in this study
with promising results, many advanced techniques remain unexplored.
Future work could investigate these more sophisticated algorithms, along
with the potential application of machine learning-based methods to
further enhance optimization processes.

The flutter phenomenon investigated in this thesis, modeled using the Vortex
Lattice aerodynamic theory coupled with a structural solver based on
modal analysis, could also be simulated through a transient
Fluid-Structure Interaction (FSI) analysis. This would allow for a
comparison of the results between the two approaches, providing further
validation and insights

Finally, the integration of Physics-Informed Neural Networks (PINNs) in
this application presents a promising avenue for future research. PINNs
are an emerging computational framework that combines data-driven
learning with governing physical laws, such as the equations of fluid
flow and structural dynamics. By embedding the physics directly into the
neural network\textquotesingle s training process, PINNs can offer a
more efficient and accurate solution to complex fluid-structure
interaction problems. Their potential to bypass some of the limitations
of traditional numerical methods, such as mesh generation and solver
convergence issues, makes them particularly attractive for engineering
systems.
