\begin{abstract}
    This thesis presents an aeroelastic analysis of a wing structure with
    the objective of optimizing its composite material to maximize flutter
    speed while minimizing mass. The study utilizes Altair's OptiStruct
    solver to compute flutter curves and assess the aeroelastic stability of
    the structure. Various optimization algorithms, implemented in Python,
    were employed to explore the design space and identify optimal
    configurations that balance structural mass and aeroelastic performance.
    Furthermore, a neural network model was developed to predict flutter
    speed based on key structural and material parameters, to see the
    potential that a surrogate model has in predicting the flutter speed.
    The results demonstrate significant improvements in both flutter speed
    and weight reduction, highlighting the potential of advanced
    optimization techniques and machine learning in aeroelastic design.
\end{abstract}

\selectlanguage{greek}
\begin{abstract}
    Η παρούσα διπλωματική εργασία παρουσιάζει μια αεροελαστική ανάλυση μιας
    πτέρυγας, με στόχο τη βελτιστοποίηση του πολυστρωματικoύ σύνθετου υλικού
    της για τη μεγιστοποίηση της ταχύτητας πτερυγισμού ελαχιστοποιώντας
    παράλληλα τη μάζα. Η μελέτη χρησιμοποιεί τον επιλυτή $OptiStruct$ της
    $Altair$ για να αξιολογήσει την αεροελαστική σταθερότητα της δομής.
    Διάφοροι αλγόριθμοι βελτιστοποίησης, που υλοποιήθηκαν στην $Python$,
    χρησιμοποιήθηκαν για να εξερευνήσουν το χώρο σχεδιασμού και να
    εντοπίσουν βέλτιστους συνδυασμούς παραμέτρων που παράγουν αποδεκτά
    χαρακτηριστικά πτερυγισμού ενώ ταυτόχρονα ελαχιστοποιούν τη μάζα της
    κατασκευής. Επιπλέον, αναπτύχθηκε ένα μοντέλο νευρωνικού δικτύου για την
    πρόβλεψη της ταχύτητας πτερυγισμού με βάση βασικές δομικές και υλικές
    παραμέτρους, για να αναδειχθούν οι δυνατότητες που έχει ένα υποκατάστατο
    μοντέλο στην πρόβλεψη της ταχύτητας πτερυγισμού. Τα αποτελέσματα
    δείχνουν σημαντικές βελτιώσεις τόσο στην ταχύτητα πτερυγισμού όσο και
    στη μείωση βάρους, τονίζοντας τις δυνατότητες των προηγμένων τεχνικών
    βελτιστοποίησης και της μηχανικής μάθησης στον αεροελαστικό σχεδιασμό.
\end{abstract}